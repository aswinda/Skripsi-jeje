%----------------------------------------------------------------------------------------
%	ABSTRACT
%----------------------------------------------------------------------------------------
\Abstract{\scriptsize 
	% ---- Tuliskan abstrak di bagian ini seperti contoh.
	Penelitian ini menganalisis sentimen masyarakat terhadap kementrian dan pendidikan di Indonesia. Untuk itu dalam pengumpulan data kata kunci “kementrian”, “pendidikan”, “sekolah”, dan “Indonesia” digunakan untuk menjaring data terkait isu ini. Penelitian ini melakukan klasifikasi orientasi sentimen dalam 3 jenis yaitu positif, negatif dan netral menggunakan. Penelitian analisis sentimen sebelumnya juga dilakukan oleh Adityawan (2014) mengenai klasifikasi \textit{Naïve Bayes} pada pesan twitter menggunakan data seimbang belum menunjukan akurasi yang cukup baik yaitu 66.42\% untuk model Multinomial dan 71.09\% untuk model Bernoulli. Untuk itu pada penelitian ini, akan dilakukan penelitian menggunakan metode yang berbeda, yaitu metode berbasis vektor untuk menganalisis apakah hasil klasifikasi metode berbasis vektor memiliki akurasi yang lebih baik dari berbasis peluang. Penelitian ini akan menganilisis hasil perbandingan antara metode klasifikasi konvensional yaitu \textit{Multinomial Naïve Bayes} dengan metode klasifikasi Rocchio. Metode Rocchio akan digunakan  untuk mengklasifikasikan data tweet, dengan menggunakan pendekatan berdasarkan kedekatan (\textit{similarity}) .
	% ---- Akhir bagian abstrak
	\normalsize}
