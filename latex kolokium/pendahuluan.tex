%----------------------------------------------------------------------------------------
%	PENDAHULUAN
%----------------------------------------------------------------------------------------
\section*{PENDAHULUAN} % Sub Judul PENDAHULUAN
% Tuliskan isi Pendahuluan di bagian bawah ini. 
% Jika ingin menambahkan Sub-Sub Judul lainnya, silakan melihat contoh yang ada.
% Sub-sub Judul 
\subsection*{Latar Belakang}
Analisis sentimen adalah bidang ilmu yang menganalisis opini, penilaian serta sentimen terhadap suatu isu tertentu \citeauthor{LIU2012} (\cite*{LIU2012}). Analisis sentimen juga dapat digunakan sebagai penentuan keputusan terhadap suatu isu atau masalah. Opini – opini yang selanjutnya akan digunakan sebagai data untuk penentuan keputusan tehadap isu yang ada. Analisis sentimen juga memegang peranan pada pengolahan opini yang mengandung polaritas, yaitu memiliki nilai sentimen yang positif ataupun negatif (Novantirani, 2014). Sosial media merupakan tempat yang memungkinkan semua orang untuk mengekspresikan opini mereka ke publik \citeauthor{LIU2012} (\cite*{LIU2012}). Menurut Semiocast, lembaga riset media sosial yang berpusat di Paris, Prancis, jumlah pemilik akun Twitter di Indonesia merupakan yang terbesar kelima di dunia, dan berada pada posisi ketiga negara yang paling aktif mengirim pesan Twitter (tweet) perhari (Tempo 2012). Banyaknya pengguna Twitter dan adanya kemudahan dalam penyampaian opini melalui media ini, maka data opini berupa tweet tersebut yang kemudian dapat menjadi peluang dan dapat dimanfaatkan sebagai bahan penilaian, tingkat kepuasan dan evaluasi (Novantirani, 2014). Hal ini mendorong beberapa instansi atau kelompok tertentu untuk mendapatkan suatu informasi terkait isu yang akan dianalisis. \newline
Opini masyarakat dari twitter inilah yang akan digunakan selanjutnya menjadi data penelitian ini. Data tweet yang kemudian akan diolah menjadi data yang mengandung sentimen. Penelitian ini melakukan analisis terhadap data tweet terkait isu mengenai kementrian dan pendidikan. Data tweet inilah yang akan digunakan sebagai bahan penelitian untuk penilaian, tingkat kepuasan serta evaluasi kinerja pemerintahan khususnya di bidang kementrian dan pendidikan. Isu inilah yang akan menjadi kata kunci pengambilan data yang akan diolah selanjutnya. Pada penelitian Institute for Development of Economics and Finance (Indef) pada tahun 2015, berhasil menjaring 12 juta tweet terkait pemerintahan dan 150 ribu diantaranya memiliki tema pembangunan (Tempo 2015). Banyaknya jumlah tweet terkait pemerintahan khususnya dibidang kementrian dan pendidikan inilah yang mendorong dilakukannya penelitian ini dengan menyertakan kata tersebut tersebut sebagai kata kunci dalam pengumpulan data. \newline
Untuk dapat mengetahui informasi, data tweet perlu diolah yang selanjutnya dilakukan klasifikasi untuk mengetahui apakah isu tersebut masuk ke dalam sentimen positif, negatif, atau netral. Proses klasifikasi ini dapat dilakukan dengan beberapa metode. Metode klasifikasi yang umumnya digunakan yaitu berbasis peluang dan berbasis vektor. Untuk klasifikasi berbasis peluang metode yang dapat digunakan diantaranya naïve bayes dengan pemodelan bernauli dan multivariant. Sedangkan untuk klasifikasi berbasis vektor, metode yang dapat digunakan diantaranya KNN dan Rocchio. \newline
Penelitian analisis sentimen sebelumnya juga dilakukan oleh \cite{ADITYAWAN2014} (\cite*{ADITYAWAN2014}) mengenai klasifikasi Naïve Bayes pada pesan twitter menggunakan data seimbang belum menunjukan akurasi yang cukup baik yaitu 66.42\% untuk model Multinomial dan 71.09\% untuk model Bernoulli. Untuk itu pada penelitian ini, akan dilakukan penelitian menggunakan metode yang berbeda, yaitu metode berbasis vektor untuk menganalisis apakah hasil klasifikasi metode berbasis vektor memiliki akurasi yang lebih baik dari berbasis peluang. \newline
Penelitian ini menggunakan metode klasifikasi Rocchio dengan pendekatan kesamaan (similarity) dan menggunakan tiga kategori sentimen yaitu netral, postif, dan negatif. Metode ini yang kemudian akan digunakan apakah data tweet masuk ke dalam sentimen positif, negatif ,atau netral. Rocchio dengan pendekatan similarity yang digunakan  karena similarity menghitung berdasarkan kedekatan dokumen. Dari sinilah selanjutnya dapat diperoleh kesimpulan data tweet yang ada terkait isu mengenai kementrian dan pendidikan mendapatkan sentimen seperti apa di kalangan masyarakat. 


% Sub-sub Judul 
\subsection*{Rumusan Masalah}
Berdasarkan latar belakang, perumusan masalah dalam penelitian ini adalah: 
\begin{enumerate}[noitemsep] 
	\item Apakah Rocchio dapat meningkatkan komputasi tanpa mengurangi akurasi?
	\item Bagaimana metode Rocchio diimplementasikan pada data twitter berbahasa Indonesia?
	\item Bagaimana perbandingan akurasi metode Rocchio dengan Multinomial naïve bayes ?
\end{enumerate}

\subsection*{Tujuan}
Tujuan penelitian ini adalah: 
\begin{enumerate}[noitemsep] 
	\item Apakah Rocchio dapat meningkatkan komputasi pada proses klasifikasi tanpa mengurangi akurasi?
	\item Bagaimana metode Rocchio diimplementasikan pada analisis sentimen twitter berbahasa Indonesia?
	\item Bagaimana perbandingan akurasi hasil klasifikasi metode Rocchio jika dibandingkan dengan nenggunakan Multinomial naïve bayes ?
\end{enumerate}

\subsection*{Manfaat}
Penelitian ini diharapkan dapat membantu entitas yang ingin mengetahui isu tertentu dari data Twitter. Penelitian ini juga diharapkan dapat memberikan informasi apakah isu tersebut mengandung sentiment positif, negative, atau netral.
