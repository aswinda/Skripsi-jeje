%----------------------------------------------------------------------------------------
%	ABSTRACT
%----------------------------------------------------------------------------------------
\Abstract{\scriptsize 
	% ---- Tuliskan abstrak di bagian ini seperti contoh.
	Berbagai opini masyarakat yang muncul di Twitter khususnya di bidang pemerintahan dan pendidikan dapat dijadikan bahan informasi untuk melihat nilai sentimen mengenai pemerintahan di masyarakat. Penelitian ini menganalisis sentimen masyarakat terhadap kementrian dan pendidikan di Indonesia. Penelitian ini melakukan klasifikasi orientasi sentimen dalam 3 jenis yaitu positif, negatif dan netral menggunakan metode klasifikasi Rocchio. Metode Rocchio akan digunakan  untuk mengklasifikasikan data tweet, dengan menggunakan pendekatan berdasarkan kedekatan (\textit{similarity}).
	% ---- Akhir bagian abstrak
	\normalsize}
